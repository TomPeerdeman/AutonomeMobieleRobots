\documentclass[a4paper]{article}

\usepackage{caption}
\usepackage{listings}
\usepackage{fancyhdr}
\usepackage[top=3cm,bottom=3cm,left=3cm,right=3cm]{geometry}
\usepackage{color}
\usepackage{amsmath}
\usepackage{graphicx}
\usepackage{tabulary}

\definecolor{dkgreen}{rgb}{0,0.6,0}
\definecolor{gray}{rgb}{0.5,0.5,0.5}
\definecolor{mauve}{rgb}{0.58,0,0.82}

\lstset{frame=tb,
  language=Octave,
  aboveskip=3mm,
  belowskip=3mm,
  showstringspaces=false,
  columns=flexible,
  basicstyle={\small\ttfamily},
  numbers=none,
  numberstyle=\tiny\color{gray},
  keywordstyle=\color{blue},
  commentstyle=\color{dkgreen},
  stringstyle=\color{mauve},
  breaklines=true,
  breakatwhitespace=true,
  tabsize=3
}

\newcommand{\HRule}{\rule{\linewidth}{0.5mm}}
\pagestyle{fancy}
\lfoot{\small \color{gray}Tom Peerdeman - 10266186}
\cfoot{\thepage}
\rfoot{\small \color{gray}Ren\'e Aparicio Sa\'ez - 10214054}
\lhead{\small \color{gray}Autonome Mobiele Robots}

\begin{document}
\begin{titlepage}
\begin{center}
\textsc{\Large Autonome Mobiele Robots}\\[0.5cm]
\HRule \\[0,4cm]
\textsc{\huge \bfseries NXT - Particle filter based simultaneous localization and mapping}
\HRule \\[8cm]
\begin{minipage}{0.4\textwidth}
\begin{flushleft}\large
\emph{Auteurs: Tom Peerdeman \& Ren\'e Aparicio Saez}\\
\end{flushleft}
\end{minipage}
\begin{minipage}{0.4\textwidth}
\begin{flushright}\large
\emph{Datum: \today\\\hspace{1cm}}\\
\end{flushright}
\end{minipage}
\end{center}
\end{titlepage}

\tableofcontents
\newpage

\section{Materiaal}
Om de experimenten uit dit rapport te kunnen uitvoeren zijn de volgende materialen gebruikt:\\
- PC/Laptop met Matlab\\
- Boek: Autonomous Mobile Robots 2th Edition - Roland Siegwart et al.\\
- NXT-Robot\\
- Logitech Webcam\\
- Gloeilamp\\
- Zwarte tape

\section{Introduction}
Een autonome mobiele robots moet een kaart kunnen opbouwen van zijn omgeving. Aan de hand van zijn eigen gemaakte kaart zou hij met behulp van bekende kaarten kunnen bepalen waar hij zich in de wereld bevindt. Een veel gebruikte methode om een kaart al rijdend op te bouwen is SLAM (Simultaneous Localization And Mapping). Er kan dan met een bepaalde zekerheid bepaald worden waar de robot zich momenteel bevindt. Het is de bedoeling dat de robot in een gebied rond kan rijden en hierbij een goede kaart kan maken. Zo moet hij bijvoorbeeld na het rijden van een rondje weer hetzelfde deel van de kaart zien (mits er niks veranderd is aan de omgeving).

\section{Deel 1, Gegeven data}
\subsection{Formules}
\subsection{Parameters bepalen}
\section{Deel 2, Eigen data}
\subsection{Offline SLAM met behulp van een dataset}
\subsection{Odometrie bepalen}
\subsection{Resultaten}
\subsection{Discussie}
\subsection{Verbeteringen}


\end{document}
