\documentclass[a4paper]{article}

\usepackage{fancyhdr}
\usepackage[top=3cm,bottom=3cm,left=3cm,right=3cm]{geometry}
\usepackage{color}
\usepackage{amsmath}

\definecolor{gray}{rgb}{0.5,0.5,0.5}
\newcommand{\HRule}{\rule{\linewidth}{0.5mm}}
\pagestyle{fancy}
\lfoot{\small \color{gray}Tom Peerdeman - 10266186}
\cfoot{\thepage}
\rfoot{\small \color{gray}Ren\'e Aparicio Sa\'ez - 10214054}
\lhead{\small \color{gray}Betrouwbaarheids Intervallen}

\begin{document}
\begin{titlepage}
\begin{center}
\textsc{\Large Autonome Mobiele Robots}\\[0.5cm]
\HRule \\[0,4cm]
\textsc{\huge \bfseries NXT - Steering}
\HRule \\[8cm]
\begin{minipage}{0.4\textwidth}
\begin{flushleft}\large
\emph{Auteurs: Tom Peerdeman \& Ren\'e Aparicio Saez}\\
\end{flushleft}
\end{minipage}
\begin{minipage}{0.4\textwidth}
\begin{flushright}\large
\emph{Datum: \today\\\hspace{1cm}}\\
\end{flushright}
\end{minipage}
\end{center}
\end{titlepage}

\section{Kinematica}
Insert plaatje uit opgave hier

\subsection{Kinematica van twee robots}
Robot met twee wielen:\\
$\mathrm{[\dot{x}\; \dot{y}\; \dot{\theta}]^T} = f\mathrm{(l, r, \theta, \varphi_{1}, \varphi_{2})}$\\
\\
De variabelen voor het castor-wiel hoeven niet meegenomen te worden in de functie, omdat het castor-wiel niet gemotoriseerd is. Omdat deze daarom geen invloed uitoefent op de richting of snelheid van het voertuig, kan het castor-wiel gezien worden als een sleepwiel dat in de functie te verwaarlozen valt.\\\\
Robot met drie wielen:\\
$\mathrm{[\dot{x}\; \dot{y}\; \dot{\theta}]^T} = f\mathrm{(l, r, \theta, \varphi_{1}, \varphi_{2}, \varphi_{3})}$\\


\subsubsection{Odometry}
a.\\
b.

\section{Steering Experimenten}

\subsection{Theorie (a)}
bla

\subsection{Feedback}
bla

\subsection{Implementatie}
bla

\end{document}